There is a wide variety of mobile social networks that are available for different platforms. These social networks range from being ubiquitous (Facebook, Instagram, Snapchat, Twitter, etc.) to incredibly niche (targeting cat lovers, etc.). After researching the rise of various large social networks, we have seen that most of them started off in the niche social network category. Facebook is a prime example of this. It is currently the largest social network in the world but it started off as only a minute target audience. In the beginning, Facebook only allowed users with a Harvard.edu email address to sign up for their social network. They then expanded and have become an incredibly successful publicly traded company.
The current trend in mobile social networks has been leaning towards finding that initial niche. This has varied from the previously mentioned animal lover groups to the athletic groups. A new trend that has started to take place over the past year is forming groups by localization. The thought process behind this is that you are bound to have common interests by the sheer nature of being in the same vicinity as the other people you are connected with. The most popular localization mobile social network out right now is Yik Yak~\cite{YikYak}. In Lehman's terms, it is an app similar to Twitter that replicates Twitter's short message feed. The way it differs is that user's cannot follow other users. The posts are either anonymous or with a creative username that doesn't have to be persistent to that user. Users only see messages (called ``Yaks") that are posted within a two-mile radius. This app was released within the past year and is currently active at over 1,000 college campuses around the US. Yik Yak was also able to secure a \$300 million valuation two weeks ago. 
Location-based social networks are not a completely new concept. The paper that we presented on was proposing a location-based mobile social network, E-Small Talker. This app was different than Yik Yak for a variety of reasons. E-Small Talker was attempting to enhance the connection between one user to another user whereas Yik Yak is for a group of users in a predefined area. E-Small Talker would connect with other users using Bluetooth and using Bloom Filters would figure out the common interests between the users. The app would then alert both users and conversation would flow smoothly. This app had a low chance of success. Since Bluetooth's range is so limited, the app would have to be widely adopted and reach critical mass to give the users the intended effect.
For our app, we looked at various successes that the larger social networks have had while also looking at how the niche networks began to gain traction. We expanded on the idea of a group of users able to interact within a given distance of each other. We changed the hard coding of 2.5 miles that Yik Yak uses and implemented our own algorithm to build around popular zones.   We also wanted in incorporate aspects of local chat that are seen in mesh networking~\cite{raniwala2005architecture}