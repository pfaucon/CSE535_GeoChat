In this work we presented GeoChat, a novel chat application for the iPhone Operating System (iOS) allowing for geolocalized chat.  We discussed some of the challenges in implementation; including the transfer of data from high-response servers to compute and processing servers, limitations on accuracy of external API's, and difficulties intrinsic in the problem of grouping people only based on location.  We also demonstrate that complex behaviors in terms of locating zones and limiting chat between users can be achieved with a relatively simple database schema.

\subsection{Future Work}
This work presents a number of problems related to clustering users based only on location information, but there are many other problems that could be solved.  One such problems is how to deal with temporal artifacts such as whether a user has data for all points in a considered timeline.  Another possible opportunity for improvement is the slicing of the timeline to create higher-resolution zones.  For example when examining location-data across all time points we find a lower density of points than if we consider these zones only at the times when they are active.  For example the ASU campus has a higher density of people (users) during the day than are present at night, ignoring the effects of time introduces significant noise into the zone determination.  Finally, we don't have a way for users to maintain their zone membership.  If a user is actively communicating in a zone that they haven't been located in recently and a new zone calculation happens the user will be dropped from the zone, despite their participation in it.
